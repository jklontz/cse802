\documentclass[12pt]{article}
 
\usepackage[margin=1in]{geometry} 
\usepackage{amsmath,amsthm,amssymb}
\usepackage{graphicx}
\usepackage{float}
\usepackage{tikz}
\usetikzlibrary{arrows,shapes,trees} % loads some tikz extensions
 
\begin{document}
 
\title{Final Project Proposal}
\author{Sunpreet Arora \& Josh Klontz
CSE 802}
 
\maketitle

The objective of this project is to build an effective classifier for a twenty class subset of the ImageNet \cite{imagenet} database.
The proposed classifier will seek to levereage a variety of modern pattern recognition techniques in order to maximize rank 5 accuracy.

Image classification has a rich history as a field within computer science, and many datasets have been proposed for advancing the state of the art including Caltech 101 \cite{caltech101}, and PASCAL VOC \cite{pascal09}.

We plan on experimenting with a variety of low-level feature feature representations including color \cite{sande10}, shape \cite{ahonen06}, and gradient information \cite{lowe04,dalal05}.
These features will be pooled into local histograms to form our base descriptor.
For this generic image classification task we are leaning in favor of skipping objected detection and localization all together and simply compute descriptors in a dense grid across the image at various scales.
We anticipate experimenting with classic dimensionality reduction techniques such as PCA \cite{turk91} and LDA \cite{belhumeur97}.
Provided the classifier is flexible enough, Spectral Hashing \cite{weiss2008} and Product Quantization \cite{jegou2011} appear to be promising recent approaches to dimensionality reduction.

\bibliographystyle{plain}
\bibliography{proposal}

\end{document}